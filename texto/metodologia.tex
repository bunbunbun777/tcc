\newpage\section{Metodologia}
\label{chap:Metodologia}


O trabalho, quanto à sua natureza, é considerado como de Pesquisa Aplicada, pois visa o tratamento de um problema concreto. Espera-se ao fim, que a própria pesquisa apresente resultados sólidos em se tratando da resolução do problema apresentado.

Quanto aos objetivos, se classifica como pesquisa exploratória, visto o objetivo de combinar práticas existentes (algoritmos auto-organizáveis) em busca da resolução de um problema.

No que se refere aos procedimentos, é considerada como pesquisa experimental, novamente por se apresentar como aplicação de métodos e técnicas. Para a realização do trabalho, se fará uso de ensaios e estudos de laboratório. Onde, a partir do simulador GRUBiX, poderão ser feitas as simulações, testes e avaliações dos algoritmos.
              
Ainda em relação às práticas metodológicas, a pesquisa também se classifica como quantitativa. Serão realizados testes, simulações e avaliações de forma quantitativa. Ao fim serão comparados os valores numéricos de cada experimento.

\subsection{Procedimentos Metodológicos}

Os experimentos realizados neste trabalho basearam-se em simulação. O uso de simulação justificou-se pela complexidade em se aplicar o algoritmo proposto em uma grande quantidade de nós sensores, questões financeiras e relacionadas ao tempo para se testar o algoritmo em uma grande variedade de cenários também foram avaliadas.

A próxima seção introduz os conceitos referentes ao simulador utilizado, bem como os conceitos de simulação orientada a eventos.

\subsubsection{\emph{GRUBiX} e Simulação Orientada a Eventos}
Para desenvolvimento do algoritmo proposto neste trabalho, foi utilizado o simulador de redes sem fio \emph{GRUBiX}, desenvolvido no Departamento de Ciência da Computação, pelo grupo de pesquisas GRUBi, da Universidade Federal de Lavras \cite{grubi}.

Este simulador é uma evolução do projeto \emph{Shox} \cite{shox}. Assim como o \emph{Shox}, o simulador \emph{GRUBiX} caracteriza-se como um ambiente de simulação orientado a eventos. De forma simplificada, em um ambiente de simulação orientado a eventos, a linha do tempo de uma simulação é representada por uma lista de eventos, em que cada novo evento é inserido nesta lista. A posição em que cada evento é inserido nesta lista varia de acordo com o momento em que cada evento deve ocorrer.

Adicionalmente, estes simuladores disponibilizam uma API\footnote{\emph{Application Programming Interface} - Interface de Programação de Aplicação.} para que se possa desenvolver toda a pilha de protocolos presente em um nó sensor sem fio. Essa disponibilidade permite que cada camada da pilha de protocolos seja personalizada de acordo com as necessidades de cada aplicação.

Assim como o \emph{Shox}, o simulador \emph{GRUBiX} utiliza a linguagem de programação Java. O uso de uma linguagem orientada a objetos, nesse caso Java, indica que as camadas da pilha de protocolos sejam modeladas como objetos. O uso dessa modelagem orientada a objetos permite que cada camada seja personalizada através de mecanismos de herança. Portanto, basta que a nova camada personalizada herde as funcionalidades de uma camada base para que se tenha a possibilidade de alterar o comportamento padrão da camada.


O desenvolvimento deste trabalho caracteriza-se por um algoritmo sobre a camada de aplicação da pilha de protocolos dos nós sensores sem fio e dos VANTs.

Os próximos capítulos discutem o funcionamento do algoritmo e os resultados apresentados.




