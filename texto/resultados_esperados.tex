\chapter{Resultados Esperados}
\label{chap:Resultados Esperados}
 
Ao fim deste trabalho espera-se o desenvolvimento de uma aplicação completa de monitoramento e detecção eficientes de intrusos em uma determinada área de interesse.
Adicionalmente, os algoritmos deverão apresentar resultados eficazes quanto a preocupações como economia de energia, latência e largura de banda para que se alcance os objetivos considerando-se as limitações e desafios de uma rede de sensores sem fio.

Deverão ser apresentados quatro principais algoritmos para a resolução do problema:

\begin{description}
	\item[ Distribuição de Ferormônio:]  Algoritmo de distribuição de ferormônio e formação de rastros do \vant dentro da rede de sensores.
	\item[ Detecção e Propagação do Alarme de Evento:] Algoritmo para a detecção e entrega do alarme de evento ao \vant mais propício.
	\item[ \emph{Tracking} e Perseguição do Intruso:]  Algoritmo para a perseguição e trilha de um evento móvel na rede.
	\item[ Deslocamento do \vant: ] Algoritmo de coordenação do \vant para que o mesmo alcance o local onde o evento de interesse ocorreu.
\end{description}

Experiências computacionais comparando a eficiência destes algoritmos serão realizadas e discutidas quanto à viabilidade e ao desempenho de cada algoritmo em diferentes cenários.

Um resultado adicional desejado, porém condicionado à limitações atuais, é a implementação concreta do sistema de monitoramento utilizando-se nós sensores reais e um \vant que ainda se encontra em desenvolvimento. Caso vencidas as limitações, e consequentemente a implementação do projeto se viabilizar, os resultados da aplicação real também serão discutidos.