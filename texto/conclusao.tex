\newpage\section{Conclusões}
\label{chap:Conclusões}

Este trabalho desenvolveu e avaliou o uso um algoritmo para a entrega de alarmes em uma rede de sensores sem fio. Foram realizados experimentos computacionais para comprovar a eficiência e eficácia da técnica desenvolvida em relação a uma técnica tradicional de entrega de alarmes em uma rede de sensores.

Utilizou-se o potencial combinado do uso de Redes de Sensores sem Fio e Veículos Aéreos não Tripulados, considerando-se as características de ambos.

Foram realizadas simulações utilizando o simulador \emph{GRUBiX} e tornou-se possível a verificação dos resultados que comprovaram a eficiência e eficácia do algoritmo em questão. As estatísticas obtidas revelam uma maior eficiência em relação ao número de mensagens utilizadas para se entregar um alarme a um \vant.

O trabalho reafirma as vantagens da utilização de nós sensores sem fio em conjunto com outras tecnologias que possibilitem apoio à eficiência da rede.

Como trabalhos futuros, podem ainda ser considerados algoritmos para:

\begin{description}
	\item[Reforço de Conectividade da Rede:] desenvolver um algoritmo para que os \vants que sobrevoam a área de interesse atuem como nós sensores móveis, de modo que os \vants possam complementar a conectividade da rede em caso de falhas em nós sensores.
	
	\item[\emph{Tracking} e Perseguição de Eventos: ] elaboração de algoritmos para a perseguição de eventos de interesse em casos em que os eventos alterem suas posições após serem detectados, visto que o algoritmo deste trabalho não prevê casos onde o evento se locomova.

	\item[Reforço dos Rastros de Ferormônio:] Em situações em que se utiliza mais de uma aeronave pode se tornar interessante desenvolver uma política de reforço dos rastros de ferormônio construídos em por cada \vant. Nesse caso, cada \vant pode reforçar o gradiente de todos os outros, desde que seus rastros se encontrem em algum ponto da área de interesse.
\end{description}
\newpage