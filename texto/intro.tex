\section{Introdução}
\label{chap:Introdução}

Uma tendência que tem ganhado força na área de redes de sensores sem fio é o uso
de nós sensores heterogêneos. Este nós podem ser utilizados como ferramentas
para se cumprir os requisitos de sofisticadas aplicações emergentes, tais como
sistemas de monitoramento \cite{Freitas20092}.

Uma maneira simples de se monitorar áreas de interesse é espalhar sensores por
toda sua extensão. Contudo, um dos maiores desafios no desenvolvimento de tais
aplicações em redes de sensores encontra-se em como prover coordenação entre os
nós envolvidos, atendendo assim, às necessidades dos usuários
\cite{Mhatre2005}.

Este trabalho apresenta a investigação de uma estratégia para coordenar um
conjunto de nós sensores terrestres estáticos (posicionados no solo) e de \uavs
(\emph{Unmanned Aerial Vehicles} - UAV) que carregam uma variedade de sensores.
Esta coordenação tem como objetivo prover monitoramento e detecção eficientes de
intrusos em uma determinada área de interesse. Preocupações como economia de
energia, latência e largura de banda são exploradas para que se alcance um
monitoramento eficiente considerando-se as limitações e desafios de uma rede de
sensores sem fio.

Dentre as estratégias utilizadas, destacam-se as técnicas de Auto-Organização
Emergente, que se apresentam como técnicas que não utilizam controles externos
ou centrais. Em sistemas auto-organizados, as entidades individuais interagem
entre si localmente. Porém, pelas interações locais, promovem um comportamento
global emergente.

Diversos nós sensores são distribuídos em uma área de interesse, bem como são
distribuídos estrategicamente \vants sobrevoando esta área. Os nós
sensores terrestres são configurados para acionar alarmes na ocorrência de um
dado evento de interesse, enquanto os
\vants recebem os alarmes e têm que decidir qual \vant é o mais hábil a tratar o
alarme acionado.

O sistema proposto por este trabalho é projetado para que os nó sensores e
\vants se comuniquem e tomem decisões autonomamente, isto é, sem nenhum controle
externo ou centralizado. Ao fim, é gerado um comportamento global emergente a
partir das pequenas interações entre os indivíduos do sistema (nós sensores e
UAVs).


%%%%%%%%%%%%%%%%%%%%%%%%%%%%%%%%%%%%%%%%%%%%%%%%%%%%%%%%%%%%%%%%%%%%%%%%%%%%%%%%
%%

\subsection{Motivação}
\wsn (RSSFs) são utilizadas para se aumentar a eficiência de uma gama de
aplicações, tais como detecção de alvos, monitoramento, vigilância ou
gerenciamento de desastres. \rssfs utilizando nós sensores estáticos têm sido
desenvolvidas, testadas e utilizadas em diversas aplicações de monitoramento
\cite{Mainwaring2002}.

Contudo, nós sensores terrestres apresentam algumas limitações, especificamente,
neste caso, em relação ao raio de comunicação de cada nó. O uso de nós sensores
móveis em tais situações pode prover melhorias significativas. Nós sensores
móveis podem prover habilidades para que a rede possa se adaptar dinamicamente
aos eventos ocorridos no ambiente, bem como colaboram para se aumentar a
conectividade dentre da rede  \cite{Aware}.

Um nó concentrador estático é geralmente localizado nas extremidades de uma
RSSF, todavia, isto geralmente requer uma longa cadeia de troca de mensagens
(\emph{multi hop}) para que um nó sensor consiga transmitir uma mensagem para o
nó concentrador. Isto resulta em baixo desempenho do sistema, uso ineficiente da
energia e desperdício de largura de banda \cite{Chang2007}.

Neste contexto, tem-se a possibilidade de utilizar \vants como nós sensores
móveis em uma RSSF. Autores como \cite{Lucchi2007} têm considerado o uso de nós
sensores terrestres espalhados por uma área de interesse. Estes nós podem
coletar diversos tipos de informação do ambiente, tais como temperatura,
pressão, umidade, etc, e possuem a capacidade de se comunicar com o \vant no
momento em que a aeronave sobrevoa as áreas onde os nós se encontram
posicionados.

O uso de \vants como sensores móveis da rede pode prover a habilidade de se
monitorar os eventos ocorridos em uma maior granularidade. Os nós sensores podem
detectar a ocorrência de um evento, porém podem não conter recursos suficientes
para análises mais detalhadas, delegando assim a ocorrência a um \vant com
habilidades específicas para tratar o problema.

Neste contexto é que o presente trabalho propõe técnicas para a coordenação de
uma \rssf heterogênea. O cenário principal é a distribuição de milhares de
sensores terrestres simples e de pouca capacidade computacional, utilizados
somente para a detecção de eventos de interesse e algumas poucas aeronaves não
tripuladas especializadas no tratamento de diferentes eventos. Detectados os
eventos, os nós sensores devem se coordenar e garantir que a mensagem seja
entregue ao \vant mais hábil para tratar o alarme de ocorrência destes eventos.


\subsection{Objetivos}

Este trabalho tem por objetivo global desenvolver e aplicar uma técnica de
coordenação entre nós sensores sem fio e \uavs para aplicações de monitoramento
e vigilância. Para o alcance deste objetivo o trabalho inspira-se em técnicas
Auto-Organizáveis para a coordenação e controle da \rssf.

\subsubsection{Objetivos Específicos}

Este trabalho tem como objetivos específicos:

\begin{description}

	\item [Pesquisa e elaboração de algoritmo de coordenação de nós
sensores e \vants:] desenvolvimento de um algoritmo para detecção e entrega de
alarmes a partir dos nós sensores. Desenvolvimento de uma heurística
bio-inspirada para localização eficiente do \uav mais próximo.

	\item [Implementação dos algoritmos e técnicas no ambiente de simulação
GRUBiX:] implementação do algoritmo no ambiente de simulação
\emph{open source} GRUBiX, desenvolvido pelo Grupo de Redes Ubíquas do
Departamento de Ciência da Computação da Universidade Federal de Lavras.
Adicionalmente, acrescentar melhorias e funcionalidades à atual versão do
simulador.

	\item [Comparação entre um algoritmo convencional e o desenvolvido no
trabalho:] comparações quantitativas dos resultados a partir de experimentos
computacionais considerando-se cada uma das técnicas em determinado cenário e configuração.

\end{description}

%%%%%%%%%%%%%%%%%%%%%%%%%%%%%%%%%%%%%%%%%%%%%%%%%%%%%%%%%%%%%%%%%%%%%%%%%%%%%%%%
%
\subsection{Organização do Trabalho}

Este trabalho encontra-se organizado em cinco capítulos. O capítulo
\ref{chap:Introdução} apresenta uma introdução, a motivação, os objetivos e
definição do problema estudado. No capítulo \ref{chap:Referencial Teórico} podem
ser encontradas as definições e bases teóricas para o entendimento do problema.
A metodologia para realização do trabalho encontra-se no capítulo
\ref{chap:Metodologia}. Os capítulos \ref{chap:Desenvolvimento} e
\ref{chap:Resultados}, respectivamente, demonstram o funcionamento da aplicação e seus resultados. Por fim, capítulo \ref{chap:Conclusões} discute a conclusões em relação ao trabalho realizado.


