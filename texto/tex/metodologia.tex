\chapter{Metodologia}
\label{chap:Metodologia}


O trabalho, quanto à sua natureza, é considerado como de Pesquisa Aplicada, pois visa o tratamento de um problema concreto. Espera-se ao fim, que a própria pesquisa apresente resultados sólidos em se tratando da resolução do problema apresentado.

Quanto aos objetivos, se classifica como pesquisa exploratória, visto o objetivo de combinar práticas existentes (algoritmos auto-organizáveis) em busca da resolução de um problema.

No que se refere aos procedimentos, é considerada como pesquisa experimental, novamente por se apresentar como aplicação de métodos e técnicas. Para a realização do trabalho, se fará uso de ensaios e estudos de laboratório. Onde, a partir do simulador GRUBiX, poderão ser feitas as simulacões, testes e avaliações dos algoritmos.
              
Ainda em relação às práticas metodológicas, a pesquisa também se classifica como quantitativa. Serão realizados testes, simulações e avaliações de forma quantitative. Ao fim serão comparados os valores numéricos de cada experimento.
